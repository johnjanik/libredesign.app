\documentclass[11pt,a4paper]{article}

\usepackage[utf8]{inputenc}
\usepackage[T1]{fontenc}
\usepackage{lmodern}
\usepackage[margin=1in]{geometry}
\usepackage{hyperref}
\usepackage{xcolor}
\usepackage{listings}
\usepackage{tcolorbox}
\usepackage{enumitem}
\usepackage{graphicx}
\usepackage{fancyhdr}
\usepackage{titlesec}

% Colors
\definecolor{anthropicpurple}{RGB}{103, 58, 183}
\definecolor{codebg}{RGB}{245, 245, 245}
\definecolor{codeframe}{RGB}{200, 200, 200}
\definecolor{linkblue}{RGB}{0, 102, 204}
\definecolor{warnorange}{RGB}{255, 152, 0}
\definecolor{successgreen}{RGB}{76, 175, 80}

% Hyperref setup
\hypersetup{
    colorlinks=true,
    linkcolor=anthropicpurple,
    urlcolor=linkblue,
    pdftitle={DesignLibre OAuth Setup Guide},
    pdfauthor={DesignLibre Team}
}

% Code listing setup
\lstset{
    basicstyle=\ttfamily\small,
    backgroundcolor=\color{codebg},
    frame=single,
    rulecolor=\color{codeframe},
    breaklines=true,
    breakatwhitespace=true,
    showstringspaces=false,
    tabsize=2,
    captionpos=b,
    numbers=left,
    numberstyle=\tiny\color{gray},
    keywordstyle=\color{anthropicpurple}\bfseries,
    stringstyle=\color{successgreen},
    commentstyle=\color{gray}\itshape
}

% Custom boxes
\tcbuselibrary{skins,breakable}

\newtcolorbox{infobox}{
    colback=blue!5!white,
    colframe=linkblue,
    title=\textbf{Info},
    fonttitle=\bfseries,
    breakable
}

\newtcolorbox{warningbox}{
    colback=orange!5!white,
    colframe=warnorange,
    title=\textbf{Warning},
    fonttitle=\bfseries,
    breakable
}

\newtcolorbox{successbox}{
    colback=green!5!white,
    colframe=successgreen,
    title=\textbf{Success},
    fonttitle=\bfseries,
    breakable
}

% Header/Footer
\pagestyle{fancy}
\fancyhf{}
\fancyhead[L]{\textcolor{anthropicpurple}{DesignLibre}}
\fancyhead[R]{OAuth Setup Guide}
\fancyfoot[C]{\thepage}
\renewcommand{\headrulewidth}{0.4pt}

% Title formatting
\titleformat{\section}
    {\Large\bfseries\color{anthropicpurple}}
    {\thesection}{1em}{}
\titleformat{\subsection}
    {\large\bfseries\color{anthropicpurple!80!black}}
    {\thesubsection}{1em}{}

\begin{document}

% Title Page
\begin{titlepage}
    \centering
    \vspace*{2cm}

    {\Huge\bfseries\textcolor{anthropicpurple}{DesignLibre}\par}
    \vspace{0.5cm}
    {\Large OAuth Authentication Setup Guide\par}

    \vspace{2cm}

    {\large Integrating Claude API with OAuth 2.0 + PKCE\par}

    \vspace{3cm}

    \begin{tcolorbox}[colback=anthropicpurple!10!white,colframe=anthropicpurple,width=0.8\textwidth]
        \centering
        This guide covers:
        \begin{itemize}[leftmargin=*]
            \item Registering your application with Anthropic
            \item Configuring OAuth client credentials
            \item Creating the OAuth callback page
            \item Testing the authentication flow
        \end{itemize}
    \end{tcolorbox}

    \vfill

    {\large Version 1.0\par}
    {\large January 2026\par}
\end{titlepage}

\tableofcontents
\newpage

% =============================================================================
\section{Introduction}
% =============================================================================

DesignLibre integrates with Claude AI through Anthropic's API. While API keys provide a simple authentication method, OAuth 2.0 with PKCE (Proof Key for Code Exchange) offers a more secure, user-friendly authentication flow---similar to how Claude Code CLI authenticates users.

\subsection{Why OAuth?}

\begin{itemize}
    \item \textbf{No API key management}: Users authenticate with their Anthropic account
    \item \textbf{Secure}: PKCE prevents authorization code interception attacks
    \item \textbf{User-friendly}: Familiar "Sign in with..." flow
    \item \textbf{Token refresh}: Automatic token renewal without re-authentication
    \item \textbf{Scoped access}: Request only the permissions needed
\end{itemize}

\subsection{Prerequisites}

Before proceeding, ensure you have:

\begin{enumerate}
    \item An Anthropic account with API access
    \item A deployed web application with HTTPS (required for OAuth)
    \item Access to your web server to add the callback page
\end{enumerate}

\begin{warningbox}
OAuth requires HTTPS in production. For local development, \texttt{localhost} is typically allowed, but production deployments must use a valid SSL certificate.
\end{warningbox}

% =============================================================================
\section{Registering Your Application with Anthropic}
% =============================================================================

\subsection{Step 1: Access the Anthropic Console}

Navigate to the Anthropic Console at:

\begin{center}
    \url{https://console.anthropic.com}
\end{center}

Sign in with your Anthropic account credentials.

\subsection{Step 2: Navigate to OAuth Applications}

\begin{enumerate}
    \item Click on \textbf{Settings} in the left sidebar
    \item Select \textbf{OAuth Applications} (or \textbf{API \& OAuth})
    \item Click \textbf{Create New Application}
\end{enumerate}

\begin{infobox}
If you don't see the OAuth Applications option, your account may need to be upgraded or you may need to request access from Anthropic. Contact \href{mailto:support@anthropic.com}{support@anthropic.com} for assistance.
\end{infobox}

\subsection{Step 3: Configure Application Details}

Fill in the application registration form:

\begin{table}[h]
\centering
\begin{tabular}{|l|p{8cm}|}
\hline
\textbf{Field} & \textbf{Value} \\
\hline
Application Name & DesignLibre \\
\hline
Description & AI-powered design tool with Claude integration \\
\hline
Application Type & Single Page Application (SPA) \\
\hline
Homepage URL & \texttt{https://your-domain.com} \\
\hline
Redirect URIs & \texttt{https://your-domain.com/auth/callback} \\
\hline
\end{tabular}
\caption{OAuth Application Configuration}
\end{table}

\subsection{Step 4: Configure Redirect URIs}

Add all redirect URIs where your application will receive OAuth callbacks:

\begin{lstlisting}[language=bash,title=Example Redirect URIs]
# Production
https://designlibre.app/auth/callback
https://www.designlibre.app/auth/callback

# Staging
https://staging.designlibre.app/auth/callback

# Local Development
http://localhost:3000/auth/callback
http://localhost:5173/auth/callback
\end{lstlisting}

\begin{warningbox}
Each environment needs its own redirect URI. The redirect URI in the OAuth request must \textbf{exactly match} one of the registered URIs---including trailing slashes and protocol.
\end{warningbox}

\subsection{Step 5: Request Scopes}

Select the OAuth scopes your application needs:

\begin{table}[h]
\centering
\begin{tabular}{|l|p{7cm}|c|}
\hline
\textbf{Scope} & \textbf{Description} & \textbf{Required} \\
\hline
\texttt{api:read} & Read access to Claude API & Yes \\
\hline
\texttt{api:write} & Write access (send messages) & Yes \\
\hline
\texttt{user:read} & Read user profile information & Optional \\
\hline
\end{tabular}
\caption{OAuth Scopes}
\end{table}

\subsection{Step 6: Obtain Client Credentials}

After creating the application, you'll receive:

\begin{itemize}
    \item \textbf{Client ID}: A public identifier for your application
    \item \textbf{Client Secret}: \textit{Not used for public clients with PKCE}
\end{itemize}

\begin{successbox}
Save your \textbf{Client ID}! You'll need it to configure the OAuth client in DesignLibre.

Example: \texttt{dl\_oauth\_abc123xyz789}
\end{successbox}

\subsection{Step 7: Update DesignLibre Configuration}

Update the OAuth client configuration in your codebase:

\begin{lstlisting}[language=JavaScript,title=src/ai/auth/oauth-client.ts]
const ANTHROPIC_OAUTH_CONFIG: OAuthConfig = {
  clientId: 'your-client-id-here',  // Replace with your Client ID
  authorizationEndpoint: 'https://console.anthropic.com/oauth/authorize',
  tokenEndpoint: 'https://console.anthropic.com/oauth/token',
  redirectUri: `${window.location.origin}/auth/callback`,
  scopes: ['api:read', 'api:write'],
};
\end{lstlisting}

% =============================================================================
\section{Creating the OAuth Callback Page}
% =============================================================================

The callback page receives the authorization code from Anthropic and posts it back to the parent window (your main application).

\subsection{Step 1: Create the Callback HTML File}

Create a new file at \texttt{/auth/callback/index.html} (or configure your router to serve this page at \texttt{/auth/callback}):

\begin{lstlisting}[language=HTML,title=public/auth/callback/index.html]
<!DOCTYPE html>
<html lang="en">
<head>
  <meta charset="UTF-8">
  <meta name="viewport" content="width=device-width, initial-scale=1.0">
  <title>Authenticating... - DesignLibre</title>
  <style>
    * {
      margin: 0;
      padding: 0;
      box-sizing: border-box;
    }

    body {
      font-family: -apple-system, BlinkMacSystemFont, 'Segoe UI',
                   Roboto, Oxygen, Ubuntu, sans-serif;
      background: linear-gradient(135deg, #667eea 0%, #764ba2 100%);
      min-height: 100vh;
      display: flex;
      align-items: center;
      justify-content: center;
    }

    .container {
      background: white;
      padding: 3rem;
      border-radius: 16px;
      box-shadow: 0 20px 60px rgba(0, 0, 0, 0.3);
      text-align: center;
      max-width: 400px;
      width: 90%;
    }

    .spinner {
      width: 50px;
      height: 50px;
      border: 4px solid #e0e0e0;
      border-top-color: #667eea;
      border-radius: 50%;
      animation: spin 1s linear infinite;
      margin: 0 auto 1.5rem;
    }

    @keyframes spin {
      to { transform: rotate(360deg); }
    }

    h1 {
      color: #333;
      font-size: 1.5rem;
      margin-bottom: 0.5rem;
    }

    p {
      color: #666;
      font-size: 0.95rem;
    }

    .error {
      background: #fee;
      border: 1px solid #fcc;
      color: #c00;
      padding: 1rem;
      border-radius: 8px;
      margin-top: 1rem;
      display: none;
    }

    .error.visible {
      display: block;
    }
  </style>
</head>
<body>
  <div class="container">
    <div class="spinner" id="spinner"></div>
    <h1 id="title">Authenticating...</h1>
    <p id="message">Please wait while we complete the sign-in process.</p>
    <div class="error" id="error"></div>
  </div>

  <script>
    (function() {
      // Parse URL parameters
      const params = new URLSearchParams(window.location.search);
      const code = params.get('code');
      const state = params.get('state');
      const error = params.get('error');
      const errorDescription = params.get('error_description');

      // UI elements
      const spinner = document.getElementById('spinner');
      const title = document.getElementById('title');
      const message = document.getElementById('message');
      const errorDiv = document.getElementById('error');

      function showError(msg) {
        spinner.style.display = 'none';
        title.textContent = 'Authentication Failed';
        message.textContent = 'Unable to complete sign-in.';
        errorDiv.textContent = msg;
        errorDiv.classList.add('visible');
      }

      function showSuccess() {
        spinner.style.display = 'none';
        title.textContent = 'Success!';
        message.textContent = 'You can close this window.';
      }

      // Handle OAuth error response
      if (error) {
        showError(errorDescription || error);
        return;
      }

      // Validate required parameters
      if (!code || !state) {
        showError('Missing authorization code or state parameter.');
        return;
      }

      // Post message to parent window
      if (window.opener) {
        try {
          window.opener.postMessage({
            type: 'oauth_callback',
            code: code,
            state: state
          }, window.location.origin);

          showSuccess();

          // Auto-close after short delay
          setTimeout(() => {
            window.close();
          }, 1500);
        } catch (err) {
          showError('Failed to communicate with the application.');
        }
      } else {
        showError('This page must be opened from DesignLibre.');
      }
    })();
  </script>
</body>
</html>
\end{lstlisting}

\subsection{Step 2: Understanding the Callback Flow}

The callback page performs these steps:

\begin{enumerate}
    \item \textbf{Parse URL Parameters}: Extract \texttt{code} and \texttt{state} from the query string
    \item \textbf{Validate State}: The state parameter prevents CSRF attacks
    \item \textbf{Post to Parent}: Send the code to the opener window via \texttt{postMessage}
    \item \textbf{Close Window}: Automatically close after successful handoff
\end{enumerate}

\begin{figure}[h]
\centering
\begin{tcolorbox}[colback=white,colframe=anthropicpurple,width=0.9\textwidth]
\begin{verbatim}
    DesignLibre App                 Anthropic Console
         |                                |
         |----[1] Open popup------------->|
         |                                |
         |                    [2] User authenticates
         |                                |
         |<---[3] Redirect to callback----|
         |       ?code=xxx&state=yyy      |
         |                                |
    Callback Page                         |
         |                                |
         |----[4] postMessage to parent   |
         |                                |
    DesignLibre App                       |
         |                                |
         |----[5] Exchange code for token-|
         |                                |
         |<---[6] Access token------------|
         |                                |
\end{verbatim}
\end{tcolorbox}
\caption{OAuth 2.0 + PKCE Flow Diagram}
\end{figure}

\subsection{Step 3: Security Considerations}

\begin{warningbox}
\textbf{Important Security Notes:}
\begin{itemize}
    \item Always validate the \texttt{origin} in \texttt{postMessage} handlers
    \item Never log or expose the authorization code
    \item Use HTTPS in production
    \item The state parameter must match what was sent in the authorization request
\end{itemize}
\end{warningbox}

\subsection{Step 4: Server Configuration}

Ensure your web server serves the callback page correctly:

\subsubsection{Nginx Configuration}

\begin{lstlisting}[language=bash,title=nginx.conf]
location /auth/callback {
    try_files $uri $uri/ /auth/callback/index.html;
}
\end{lstlisting}

\subsubsection{Apache Configuration}

\begin{lstlisting}[language=bash,title=.htaccess]
RewriteEngine On
RewriteRule ^auth/callback$ /auth/callback/index.html [L]
\end{lstlisting}

\subsubsection{Vite/SPA Configuration}

For single-page applications using Vite:

\begin{lstlisting}[language=JavaScript,title=vite.config.ts]
export default defineConfig({
  // ... other config
  build: {
    rollupOptions: {
      input: {
        main: resolve(__dirname, 'index.html'),
        callback: resolve(__dirname, 'auth/callback/index.html'),
      },
    },
  },
});
\end{lstlisting}

% =============================================================================
\section{Testing the Authentication Flow}
% =============================================================================

\subsection{Step 1: Local Development Testing}

\begin{enumerate}
    \item Start your development server
    \item Open the AI Assistant panel
    \item Click "Sign in with Claude"
    \item Verify the popup opens to Anthropic's authorization page
    \item Complete authentication
    \item Verify the callback page appears briefly
    \item Verify the popup closes automatically
    \item Check that the settings panel shows "Connected"
\end{enumerate}

\subsection{Step 2: Verify Token Storage}

Open your browser's Developer Tools and check localStorage:

\begin{lstlisting}[language=JavaScript,title=Browser Console]
// Check if tokens are stored
const tokenData = localStorage.getItem('designlibre:auth:anthropic');
if (tokenData) {
  const tokens = JSON.parse(atob(tokenData));
  console.log('Access Token:', tokens.accessToken ? 'Present' : 'Missing');
  console.log('Refresh Token:', tokens.refreshToken ? 'Present' : 'Missing');
  console.log('Expires At:', new Date(tokens.expiresAt));
}
\end{lstlisting}

\subsection{Step 3: Test API Calls}

Verify that API calls use the OAuth token:

\begin{enumerate}
    \item Open Network tab in Developer Tools
    \item Send a message in the AI Assistant
    \item Find the request to \texttt{api.anthropic.com}
    \item Verify the \texttt{Authorization: Bearer ...} header is present
\end{enumerate}

\subsection{Step 4: Test Token Refresh}

To test token refresh:

\begin{lstlisting}[language=JavaScript,title=Browser Console - Force Token Expiry]
// Manually expire the token for testing
const tokenData = localStorage.getItem('designlibre:auth:anthropic');
if (tokenData) {
  const tokens = JSON.parse(atob(tokenData));
  tokens.expiresAt = Date.now() - 1000; // Set to past
  localStorage.setItem(
    'designlibre:auth:anthropic',
    btoa(JSON.stringify(tokens))
  );
  console.log('Token marked as expired. Next API call will trigger refresh.');
}
\end{lstlisting}

% =============================================================================
\section{Troubleshooting}
% =============================================================================

\subsection{Common Issues}

\begin{table}[h]
\centering
\begin{tabular}{|p{4cm}|p{8cm}|}
\hline
\textbf{Issue} & \textbf{Solution} \\
\hline
"Invalid redirect URI" & Ensure the redirect URI exactly matches what's registered in Anthropic Console, including protocol and trailing slashes \\
\hline
"Invalid client ID" & Verify the client ID in \texttt{oauth-client.ts} matches the one from Anthropic Console \\
\hline
Popup blocked & Users may need to allow popups for your domain \\
\hline
"State mismatch" & Clear session storage and try again; may indicate CSRF attempt \\
\hline
Token refresh fails & The refresh token may have expired; user needs to re-authenticate \\
\hline
CORS errors & Ensure your domain is allowed in Anthropic's OAuth app settings \\
\hline
\end{tabular}
\caption{Troubleshooting Guide}
\end{table}

\subsection{Debug Mode}

Enable debug logging by adding to your OAuth client:

\begin{lstlisting}[language=JavaScript,title=Debug Logging]
// In oauth-client.ts, add to startAuth():
console.log('[OAuth] Starting auth flow', {
  clientId: this.config.clientId,
  redirectUri: this.config.redirectUri,
  scopes: this.config.scopes,
});

// In handleCallback():
console.log('[OAuth] Received callback', { code: '***', state });
console.log('[OAuth] Token exchange successful', {
  hasAccessToken: !!tokens.accessToken,
  hasRefreshToken: !!tokens.refreshToken,
  expiresAt: tokens.expiresAt,
});
\end{lstlisting}

% =============================================================================
\section{Production Checklist}
% =============================================================================

Before deploying to production, verify:

\begin{enumerate}[label=$\square$]
    \item OAuth application registered with Anthropic
    \item Production redirect URI added to allowed URIs
    \item HTTPS configured and working
    \item Callback page deployed and accessible
    \item Client ID updated in production config
    \item Token storage working correctly
    \item Error handling tested
    \item Token refresh tested
    \item Sign out functionality working
    \item Analytics/monitoring in place (optional)
\end{enumerate}

% =============================================================================
\section{Appendix: Complete OAuth Client Code}
% =============================================================================

For reference, here is the complete OAuth client implementation:

\begin{lstlisting}[language=JavaScript,title=src/ai/auth/oauth-client.ts (excerpt)]
export class AnthropicOAuthClient {
  private config: OAuthConfig;
  private authWindow: Window | null = null;

  constructor(config: Partial<OAuthConfig> = {}) {
    this.config = { ...ANTHROPIC_OAUTH_CONFIG, ...config };
  }

  async startAuth(): Promise<void> {
    const state = generateRandomString(32);
    const verifier = generateCodeVerifier();
    const challenge = await generateCodeChallenge(verifier);

    sessionStorage.setItem(STATE_STORAGE_KEY, state);
    sessionStorage.setItem(VERIFIER_STORAGE_KEY, verifier);

    const params = new URLSearchParams({
      response_type: 'code',
      client_id: this.config.clientId,
      redirect_uri: this.config.redirectUri,
      scope: this.config.scopes.join(' '),
      state,
      code_challenge: challenge,
      code_challenge_method: 'S256',
    });

    const authUrl = `${this.config.authorizationEndpoint}?${params}`;

    // Open centered popup
    const width = 500, height = 700;
    const left = window.screenX + (window.outerWidth - width) / 2;
    const top = window.screenY + (window.outerHeight - height) / 2;

    this.authWindow = window.open(
      authUrl,
      'Claude Authentication',
      `width=${width},height=${height},left=${left},top=${top}`
    );

    return this.waitForCallback();
  }

  // ... rest of implementation
}
\end{lstlisting}

\vspace{2cm}

\begin{center}
\textcolor{anthropicpurple}{\rule{0.5\textwidth}{1pt}}

\vspace{0.5cm}

{\large\textbf{Questions or Issues?}}

\vspace{0.3cm}

Open an issue at:\\
\url{https://github.com/johnjanik/libredesign.app/issues}

\vspace{0.5cm}

\textcolor{anthropicpurple}{\rule{0.5\textwidth}{1pt}}
\end{center}

\end{document}
