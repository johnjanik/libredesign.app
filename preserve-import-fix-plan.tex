\documentclass[12pt,a4paper]{article}
\usepackage[utf8]{inputenc}
\usepackage[T1]{fontenc}
\usepackage{geometry}
\usepackage{xcolor}
\usepackage{listings}
\usepackage{hyperref}
\usepackage{booktabs}
\usepackage{fancyhdr}
\usepackage{enumitem}

\geometry{margin=1in}

\definecolor{codegreen}{rgb}{0,0.6,0}
\definecolor{codegray}{rgb}{0.5,0.5,0.5}
\definecolor{codepurple}{rgb}{0.58,0,0.82}
\definecolor{backcolour}{rgb}{0.95,0.95,0.92}
\definecolor{errorred}{RGB}{200,50,50}

\lstdefinestyle{mystyle}{
    backgroundcolor=\color{backcolour},
    commentstyle=\color{codegreen},
    keywordstyle=\color{codepurple},
    numberstyle=\tiny\color{codegray},
    stringstyle=\color{codegreen},
    basicstyle=\ttfamily\footnotesize,
    breakatwhitespace=false,
    breaklines=true,
    captionpos=b,
    keepspaces=true,
    numbers=left,
    numbersep=5pt,
    showspaces=false,
    showstringspaces=false,
    showtabs=false,
    tabsize=2,
    frame=single
}
\lstset{style=mystyle}

\pagestyle{fancy}
\fancyhf{}
\rhead{DesignLibre .preserve Import Fix Plan}
\lhead{Diagnostic Analysis}
\rfoot{Page \thepage}

\title{\textbf{Diagnostic Analysis \& Fix Plan}\\
\large .preserve File Import Issue\\
\vspace{0.5cm}
\normalsize Why Colors Are Not Being Applied}

\author{Analysis by Claude AI}
\date{January 3, 2026}

\begin{document}

\maketitle
\tableofcontents
\newpage

% =============================================================================
\section{Problem Summary}
% =============================================================================

\subsection{Observed Symptoms}

When opening the \texttt{presets-screen.preserve} file in DesignLibre:

\begin{enumerate}
    \item \textbf{All frames appear white} --- The iPhone frame and all sub-frames display as \texttt{\#FFFFFF} instead of the specified colors (black background, dark grays, etc.)
    \item \textbf{Structure is correct} --- The layer hierarchy is properly imported (iPhone 16 Pro, Content Area, etc.)
    \item \textbf{Shapes are recognized} --- Pill-shaped objects, rectangles, and circles appear with correct corner radii
    \item \textbf{Text content is missing/default} --- Text styling (colors, fonts) not being applied
\end{enumerate}

\subsection{Root Cause Identification}

After analyzing the codebase, I have identified the root cause:

\begin{center}
\fcolorbox{errorred}{white}{%
\parbox{0.9\textwidth}{%
\textbf{The node factory functions do not accept appearance properties (fills, strokes, effects) as options.} When importing .preserve files, the appearance data is extracted but then discarded because the factory functions ignore these properties.
}%
}
\end{center}

% =============================================================================
\section{Technical Analysis}
% =============================================================================

\subsection{Data Flow During Import}

The import process follows this path:

\begin{enumerate}
    \item \texttt{loadPreserve(file)} reads the .preserve ZIP archive
    \item \texttt{readPreserveArchive()} parses JSON files into \texttt{PreserveArchive}
    \item For each page, \texttt{importPreserveNode()} is called
    \item \texttt{preserveToNode()} converts \texttt{PreserveNode} to \texttt{Partial<NodeData>}
    \item \texttt{sceneGraph.createNode()} creates the node with factory function
    \item \textcolor{errorred}{\textbf{Appearance properties are lost here!}}
\end{enumerate}

\subsection{The Factory Function Issue}

\subsubsection{File: \texttt{src/scene/nodes/factory.ts}}

The \texttt{createFrame} function (lines 96-118):

\begin{lstlisting}[language=TypeScript]
export interface CreateFrameOptions {
  id?: NodeId;
  name?: string;
  x?: number;
  y?: number;
  width?: number;
  height?: number;
  cornerRadius?: number;
  // NOTE: No fills, strokes, effects, opacity!
}

export function createFrame(options: CreateFrameOptions = {}): FrameNodeData {
  return {
    id: options.id ?? generateNodeId(),
    type: 'FRAME',
    name: options.name ?? getDefaultNodeName('FRAME'),
    ...DEFAULT_TRANSFORM,
    x: options.x ?? 0,
    y: options.y ?? 0,
    width: options.width ?? 100,
    height: options.height ?? 100,
    ...DEFAULT_APPEARANCE,
    fills: [solidPaint(rgba(1, 1, 1, 1))],  // <-- ALWAYS WHITE!
    constraints: DEFAULT_CONSTRAINTS,
    cornerRadius: options.cornerRadius ?? 0,
  };
}
\end{lstlisting}

\textbf{Problem:} Line 112 \textit{always} sets fills to white, ignoring any fills passed in options.

\subsubsection{Contrast with \texttt{createVector}}

The vector factory \textit{does} accept appearance options:

\begin{lstlisting}[language=TypeScript]
export interface CreateVectorOptions {
  id?: NodeId;
  name?: string;
  x?: number;
  y?: number;
  width?: number;
  height?: number;
  vectorPaths?: VectorPath[];
  fills?: Paint[];      // <-- Accepted!
  strokes?: Paint[];    // <-- Accepted!
  strokeWeight?: number;
}

export function createVector(options: CreateVectorOptions = {}): VectorNodeData {
  return {
    ...
    fills: options.fills ?? [solidPaint(rgba(0.85, 0.85, 0.85, 1))],
    strokes: options.strokes ?? [],
    strokeWeight: options.strokeWeight ?? 1,
    ...
  };
}
\end{lstlisting}

\textbf{Key difference:} Vector uses \texttt{options.fills ?? default}, while Frame hardcodes the fill.

\subsection{The Converter Output}

The \texttt{preserveToNode()} function in \texttt{node-converter.ts} correctly extracts appearance:

\begin{lstlisting}[language=TypeScript]
const getSceneProps = (node) => ({
  x: node.transform.x,
  y: node.transform.y,
  width: node.transform.width,
  height: node.transform.height,
  rotation: node.transform.rotation,
  opacity: node.appearance?.opacity ?? 1,
  fills: node.appearance?.fills?.map(preserveToPaint) ?? [],
  strokes: node.appearance?.strokes?.map(preserveToPaint) ?? [],
  strokeWeight: node.appearance?.strokeWeight,
  cornerRadius: node.appearance?.cornerRadius,
  effects: node.appearance?.effects?.map(preserveToEffect) ?? [],
});
\end{lstlisting}

This returns the correct fills from the .preserve file, but they are discarded by \texttt{createFrame()}.

\subsection{Affected Node Types}

\begin{table}[h]
\centering
\begin{tabular}{@{}llll@{}}
\toprule
\textbf{Node Type} & \textbf{Accepts fills?} & \textbf{Accepts strokes?} & \textbf{Import Works?} \\
\midrule
FRAME & \textcolor{errorred}{No} & \textcolor{errorred}{No} & \textcolor{errorred}{No} \\
GROUP & \textcolor{errorred}{No} & \textcolor{errorred}{No} & \textcolor{errorred}{No} \\
VECTOR & \textcolor{codegreen}{Yes} & \textcolor{codegreen}{Yes} & \textcolor{codegreen}{Yes} \\
TEXT & \textcolor{errorred}{No} & \textcolor{errorred}{No} & \textcolor{errorred}{No} \\
IMAGE & \textcolor{errorred}{No} & \textcolor{errorred}{No} & Partial \\
COMPONENT & \textcolor{errorred}{No} & \textcolor{errorred}{No} & \textcolor{errorred}{No} \\
INSTANCE & \textcolor{errorred}{No} & \textcolor{errorred}{No} & \textcolor{errorred}{No} \\
\bottomrule
\end{tabular}
\caption{Factory Function Appearance Support by Node Type}
\end{table}

% =============================================================================
\section{Proposed Solutions}
% =============================================================================

\subsection{Solution 1: Fix Factory Functions (Recommended)}

Update all factory functions to accept appearance properties:

\begin{lstlisting}[language=TypeScript]
export interface CreateFrameOptions {
  id?: NodeId;
  name?: string;
  x?: number;
  y?: number;
  width?: number;
  height?: number;
  cornerRadius?: number;
  // ADD THESE:
  fills?: Paint[];
  strokes?: Paint[];
  strokeWeight?: number;
  strokeAlign?: 'INSIDE' | 'CENTER' | 'OUTSIDE';
  opacity?: number;
  blendMode?: BlendMode;
  effects?: Effect[];
}

export function createFrame(options: CreateFrameOptions = {}): FrameNodeData {
  return {
    ...
    ...DEFAULT_APPEARANCE,
    fills: options.fills ?? [solidPaint(rgba(1, 1, 1, 1))],
    strokes: options.strokes ?? [],
    strokeWeight: options.strokeWeight ?? 1,
    opacity: options.opacity ?? 1,
    effects: options.effects ?? [],
    ...
  };
}
\end{lstlisting}

\textbf{Files to modify:}
\begin{itemize}
    \item \texttt{src/scene/nodes/factory.ts} --- Add appearance options to all factory functions
\end{itemize}

\textbf{Estimated changes:} $\sim$50-80 lines

\subsection{Solution 2: Post-Creation Update in Runtime}

Modify \texttt{importPreserveNode()} to update nodes after creation:

\begin{lstlisting}[language=TypeScript]
private importPreserveNode(preserveNode: PreserveNode, parentId: NodeId): NodeId {
  const nodeData = preserveToNode(preserveNode);

  const nodeId = this.sceneGraph.createNode(
    preserveNode.type as ...,
    parentId,
    -1,
    nodeData as ...
  );

  // NEW: Apply appearance properties that factory ignores
  if (preserveNode.appearance) {
    this.sceneGraph.updateNode(nodeId, {
      fills: nodeData.fills,
      strokes: nodeData.strokes,
      strokeWeight: nodeData.strokeWeight,
      opacity: nodeData.opacity,
      effects: nodeData.effects,
      cornerRadius: nodeData.cornerRadius,
    });
  }

  // Import children recursively
  ...
  return nodeId;
}
\end{lstlisting}

\textbf{Files to modify:}
\begin{itemize}
    \item \texttt{src/runtime/designlibre-runtime.ts} --- Update \texttt{importPreserveNode()}
\end{itemize}

\textbf{Estimated changes:} $\sim$15-20 lines

\subsection{Solution 3: Alternative .preserve Structure}

Since VECTOR nodes correctly import fills, one workaround is to restructure the design:

\begin{itemize}
    \item Use VECTOR nodes with rectangle paths for colored backgrounds
    \item Overlay frames for layout/children containment
    \item This is a \textbf{workaround}, not a proper fix
\end{itemize}

\textbf{Not recommended} --- creates unnecessary complexity in design files.

% =============================================================================
\section{Implementation Plan}
% =============================================================================

\subsection{Phase 1: Factory Function Updates}

\begin{enumerate}
    \item Open \texttt{src/scene/nodes/factory.ts}
    \item For each factory function (\texttt{createFrame}, \texttt{createGroup}, \texttt{createText}, \texttt{createComponent}, \texttt{createInstance}):
    \begin{enumerate}
        \item Add appearance properties to the options interface
        \item Use \texttt{options.fills ?? default} pattern instead of hardcoded values
        \item Apply the same for strokes, effects, opacity, etc.
    \end{enumerate}
    \item Update \texttt{createText} to accept \texttt{textStyles} option
\end{enumerate}

\subsection{Phase 2: Text Styling Support}

The \texttt{createText} function needs additional updates for text styles:

\begin{lstlisting}[language=TypeScript]
export interface CreateTextOptions {
  id?: NodeId;
  name?: string;
  x?: number;
  y?: number;
  width?: number;
  height?: number;
  characters?: string;
  // ADD:
  textStyles?: TextStyleRange[];
  textAlignHorizontal?: 'LEFT' | 'CENTER' | 'RIGHT' | 'JUSTIFIED';
  textAlignVertical?: 'TOP' | 'CENTER' | 'BOTTOM';
  fills?: Paint[];
}
\end{lstlisting}

\subsection{Phase 3: Testing}

\begin{enumerate}
    \item Create test .preserve file with:
    \begin{itemize}
        \item Frame with black fill
        \item Text with custom font/color
        \item Nested frames with various colors
        \item Vector with custom fills/strokes
    \end{itemize}
    \item Import the file
    \item Verify all appearance properties are applied
    \item Export and re-import to verify round-trip fidelity
\end{enumerate}

% =============================================================================
\section{Updated .preserve File Strategy}
% =============================================================================

Once the factory functions are fixed, the current \texttt{presets-screen.preserve} file should work correctly. The JSON structure is valid:

\begin{lstlisting}[language=json]
{
  "id": "iphone-frame",
  "type": "FRAME",
  "name": "iPhone 16 Pro",
  "transform": {
    "x": 0, "y": 0, "width": 393, "height": 852, "rotation": 0
  },
  "appearance": {
    "fills": [{
      "type": "SOLID",
      "color": { "r": 0, "g": 0, "b": 0, "a": 1 },
      "opacity": 1,
      "visible": true
    }],
    "cornerRadius": 0
  },
  "clipContent": true,
  "children": [...]
}
\end{lstlisting}

The appearance block is correctly structured. The issue is purely in the import/factory code.

% =============================================================================
\section{Immediate Workaround}
% =============================================================================

Until the codebase is fixed, you can manually apply colors after import:

\begin{enumerate}
    \item Open the imported .preserve file in DesignLibre
    \item Select each frame that should have a different color
    \item Use the Fill panel to change from white to the correct color
    \item Refer to the color specifications in the original documentation
\end{enumerate}

\subsection{Color Reference for Manual Application}

\begin{table}[h]
\centering
\begin{tabular}{@{}lll@{}}
\toprule
\textbf{Element} & \textbf{Current} & \textbf{Should Be} \\
\midrule
iPhone Frame & \#FFFFFF & \#000000 \\
Search Bar & \#FFFFFF & \#1C1C1E \\
Tab Selected & \#FFFFFF & \#333333 \\
Preset Card & \#FFFFFF & \#1A1A1A \\
Play Now Button & \#FFFFFF & \#262626 \\
Bottom Tab Bar & \#FFFFFF & \#141414 \\
Pink Noise Button & \#FFFFFF & \#618538 \\
\bottomrule
\end{tabular}
\caption{Manual Color Corrections Required}
\end{table}

% =============================================================================
\section{Conclusion}
% =============================================================================

The .preserve file I created is \textbf{structurally correct} and follows the specification properly. The issue lies in the DesignLibre codebase's import functionality:

\begin{itemize}
    \item The factory functions don't accept appearance properties
    \item This causes all imported frames to default to white
    \item Text nodes also lose their styling
\end{itemize}

\textbf{Recommended action:} Implement Solution 1 (update factory functions) for a permanent fix that will enable proper .preserve file import with full appearance fidelity.

The fix is straightforward and localized to \texttt{src/scene/nodes/factory.ts}. Once implemented, the existing \texttt{presets-screen.preserve} file should render correctly without modification.

\end{document}
